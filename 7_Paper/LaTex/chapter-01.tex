\chapter{绪论}\label{preface}

\section{研究背景}

当今阶段,无人机(Unmanned Aerial Vehicle)技术迅速发展,在单架无人机上配备的众多系统使单机已经能够执行各种各样的任务。
单架无人机已经可以在自然灾害的搜救、对高空电力线路的检查、日常交通及场所的安全巡查、农业播种等场景发挥出巨大的作用。
但是,单架无人机由于其运算资源有限、可搭载设备有限,在面对多种类的需求和复杂不可控的环境时,可能无法发挥出其作用;
%但是,面对复杂的应用环境和多样化的需求,单架无人机受自身软硬件条件的限制,仍然具有一些局限性;
%为了弥补单架无人机的局限性,由多架相同或不同型号的无人机组成多无人机系统,即无人机集群,协同定位,共同完成任务;
为了弥补单架无人机的局限性,可以使用集群无人机的方案,即由若干架相同型号或根据任务需求搭配不同型号的无人机组成一个多无人机系统,进行协同定位,并借此完成任务。

%通过集群的方式,能最大地发挥无人机的优势,又能避免由于单架无人机执行任务不佳或失败造成的不良后果,提高任务执行效率,扩展新的任务执行方式,从而达到提高系统可靠性,增强任务执行效果的目的。

集群是使无人机的优势较大化的一种手段,又可以达到有效降低单架无人机失误率和事故率,有效提升无人机在实际应用中的执行效率,大幅度拓展其任务执行方式,由此提升多无人机系统鲁棒性,有效提升任务完成率。对于无人机的自主导航,能够在进入未知的环境时掌握无人机的位置和姿态是使其成功的关键。尽管GPS对于掌握无人机的位置有巨大的帮助,但仍存在普适性有限和准确度不高的问题;在一些特定场景下,比如室内狭小空间,对定位精度要求很高,GPS定位的局限性就被显露出来。而SLAM技术则可以仅通过自身携带的传感器,来完成这一任务,同时达到一定的精度;

SLAM技术,即同时完成定位与建图的任务,已有较为系统的理论和研究;SLAM于20世纪90年代左右由 Smith、Cheeseman 和Self提出。
%SLAM指的是机器人在未知环境中从一个未知位置开始移动,在移动过程中根据位置和地图进行自身定位,同时在自身定位的基础上建造增量式地图,实现机器人的自主定位和导航。由于其重要的理论与应用价值,被很多学者认为是实现真正全自主移动机器人的关键\cite{1}。
SLAM是指将机器人放置在陌生环境中,令其从未知位置开始运动,在行进中根据解算出的当前位置和初步构建的地图获取自身位置信息,基于此建立增量式地图,从而达成使机器人获得自主定位和导航能力的目标\cite{1}。



但是单平台SLAM受到传感器性能的限制,存在两点不足:一是测量距离受限,单平台常用的传感器如激光雷达,其最远有效距离为200米,不能够满足大场景定位建图的需要,任务效率比较低;二是单平台构建出的三维点云相对稀疏,不能表现出足够的场景信息[2] 。

因此集群无人机协同SLAM方案有望解决单机存在的制约问题,主要表现在两个方面:一是多机意味着多传感器,能在大范围场景进行同时定位与建图;二是多机协同SLAM可以通过建图覆盖的检测,构建更加稠密和精细的点云地图。

目前单机同时定位与建图已经相对成熟,但是多机SLAM由于其控制复杂、数据传输量大、信息处理速度受限、关键数据融合效率低等问题,仍然需要大量的理论研究和实验。

\section{研究内容及论文结构}
本文研究目标旨在实现一套能够在室内高精度环境或GPS拒止环境下使用视觉进行多机定位和大范围建图的多无人机协同SLAM的方案;其中:

\begin{enumerate}
	\item 在SLAM方面:掌握一些优秀的开源方案,选择各自优点做出一定的融合。并且有一套针对地图融合的方法。
	\item 在仿真方面:在ROS的gazebo仿真平台中实现一定的集群控制方法,能够控制多个无人机协同完成同时定位与建图的任务。
	\item 在真机方面:实现单机的视觉SLAM;在安全的前提下实现双机协同SLAM,将仿真环境下的协同SLAM算法在真机上完成验证,得到场景地图
\end{enumerate}

本研究内容是多机协作进行定位与建图,多机协同SLAM能大大提高任务进行的效率,但由于无人机数量多,协同上存在一定困难;研究内容分为三个模块:SLAM模块、仿真模块和真机模块。

SLAM模块的主要内容是实现一套可协作的SLAM方案,实现的步骤有:

\begin{enumerate}
	\item 研究传统的视觉SLAM的特征点提取、匹配、初始化、后端优化等技术,研究机器人的位姿估计技术;研究并了解SLAM技术的整体框架
	\item 研究CCM-SLAM方案,重点研究其协同的机制和方法,服务端到子端的信息传递和接口设计等
	\item 研究VINS-Fusion方案中的VIO方法,研究如何利用IMU与相机数据联合进行更加准确的位姿估计
\end{enumerate}

仿真模块的主要内容是在ROS的gazebo中研究如何实现多机协同的同时定位与建图,实现的步骤有:

\begin{enumerate}
	\item 首先研究PX4和MAVROS之间的通信方式,ROS的话题发布和订阅方式,研究如何用程序解锁一架无人机、使其进入Offboard模式、起飞悬停并降落
	\item 研究如何用程序发布话题,控制无人机按照航路点飞行
	\item 研究如何构建多机的仿真环境,如何对多机进行控制,其控制策略的选择,即集中式或分布式的多机编队控制
	\item 研究如何更改无人机的定位设置,将其从GPS定位改为视觉SLAM定位;并且完成单机的摄像头内容读取
	\item 研究如何在gazebo中载入其他场景,在场景中控制无人机飞行,并且对拍摄到的画面进行建图,完成自身定位
	\item 研究如何在gazebo中完成多机基于视觉的同时定位与建图,并且拼合地图,用第三方软件显示;研究多机的联合优化与协同方法
\end{enumerate}

真机模块的主要内容是控制无人机的协同飞行及通信,实现的步骤有:

\begin{enumerate}
	\item 研究无人机通过MAVROS,MAVLINK与地面站的通信方法,尤其是用于SLAM的关键数据的传输
	\item 研究多无人机与地面站之间的、多无人机之间的数据传输
	\item 研究多无人机之间的可变基线控制技术,如何设计一个详细的算法控制基线距离
\end{enumerate}

基于研究内容的层级关系,本论文的结构主要由四层构成,如图\ref{fig1}所示。
\vspace{10pt}
\begin{figure}[!ht]
\centering
\includegraphics[width=0.8\textwidth]{structure.png}
\caption{论文结构 }
\label{fig1}
\end{figure}


第\ref{preface}章:主要介绍本文相关研究背景和现状;根据现阶段存在的问题,引出本文的研究方向和内容;从宏观的角度概括本文的内容,并且对问题做出概述。

第\ref{introduction}章:主要介绍研究中用到的ROS(Robotics Operating System,机器人操作系统)和PX4 AutoPilot飞控软件系统。

第\ref{System Overview}章:主要介绍SLAM的原理和作用,SLAM系统的基本流程,优秀的开源SLAM方案和地图融合的设计。

第\ref{Simulation}章:主要介绍仿真实验的情况,在ROS的gazebo环境中进行。

第\ref{experiment}章:主要介绍真机实验的情况,并且做出相应的评估。

第\ref{conclusion}章:总结本文研究中的创新点和有待改进之处,提出进一步研究的大致方向,展望未来的研究工作。

