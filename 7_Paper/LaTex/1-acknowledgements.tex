

\renewcommand{\baselinestretch}{1.5}
\fontsize{12pt}{13pt}\selectfont

\chapter*{致~~~~谢}
\addcontentsline{toc}{chapter}{致谢}

首先要感谢我的指导老师布树辉老师。感谢布老师在我特殊的为期接近两年的毕设进程中的指导和鼓励,从最初的C++和Opencv学习到后面的每周的组会,布老师为我指出了很多问题,同时也提出了很多醍醐灌顶的建议,感谢布老师很多次教给我的做事情先化繁为简,再由简到繁的方法;感谢布老师努力经营的Gitee平台仓库,里面涵盖了许多学习资料和实践指导,在我学习的路上提供了一条捷径;感谢布老师为我们提供的团队毕设的环境,这种每个人共同为一个目标做出贡献的感觉让我感觉自己处在一个大家庭中;布老师总告诉我实现一个目标要先实现它大概的样子,而不是纠结于一些细枝末节,在大致实现的过程中建立感性认识,之后再一点一点积累知识和方法,才能建立理性认识,这个观点让我受益匪浅。回想起来,我可能是同年级组里最早接触布老师的学生,当时老师机器学习课上30分钟学会python的方法和后来为了提前接触科研的联系,都让我明白学知识、做科研的过程中,方法很重要、有的放矢很重要,但更重要的是自律的态度、鲜明的任务规划、及时的反馈机制、对待任务的严谨,还有最重要的热爱。

还要感谢我的师兄杜万闪和李白杨,他们在我学习飞控和PX4、MAVROS相关中为我提供了莫大的帮助和环境,李白杨师兄还耐心地为我复现出现的bug并且带着我解决;感谢PI-LAB的王禹师兄,在我解决PX4的bug上提供了很大帮助。感谢我的同组同学张一竹和贾旋,他们对团队的贡献和认真的态度一直在激励着我;感谢我的室友陈泽帅为我提供了舒适的开发环境;感谢整个飞控二班在毕设的日子里愉快的氛围,Auld Lang Syne。

感谢PX4的开发团队,耐心回复我的issue;感谢stackoverflow上的程序员们,感谢ROS论坛、古月居、ORB-SLAM2和CCM-SLAM的作者、XTDrone平台。

最后要感谢我的父母对我一如既往的关心。感谢我的女朋友,遇到你是我大学生活里最幸运的事。
和你在一起的600多天才让我体会到什么是真正的生活,什么是一个真正的富有情感的人,是你让我从一开始放不开的人逐渐变成了可以对一个人完全敞开的人,是你让我以后的生活有了明确的目标和期待,更重要的是有了和我一起期待未来的人。写到致谢时才回过头看了看这四年的自己,才发现大学四年即将结束,希望自己能做一个带着回忆又勇敢向前追逐的人,江头潮平,但斯人会归。
