
\renewcommand{\baselinestretch}{1.5}
\fontsize{12pt}{13pt}\selectfont

\chapter{总结与展望} \label{conclusion}


\section{全文总结}

SLAM技术是当今机器人以及无人系统在进入未知环境时进行运动决策和场景感知的关键技术,而可协同的SLAM方案则为集群机器人或无人系统提供了更多可能。本文主要研究了SLAM技术的原理及地图拼合的技术,基于gazebo仿真平台实现对单机和多机的视觉SLAM软件在环仿真。本文主要做的工作如下:

\begin{enumerate}
	\item 简述ROS话题和服务两种通信机制和gazebo仿真的概念及配置。
	\item 简述PX4 AutoPilot软件的使用,其Failsafe机制和EKF2的更改以及飞行模式,使用MAVROS进行通信的离板程序控制。
	\item 进行SLAM系统的学习,分为四个部分;第一是了解SLAM的分类,并清晰研究使用的视觉SLAM;第二是根据相机成像原理,对其内参外参矩阵进行推导;第三是学习视觉SLAM的主要步骤,最后是推到对极几何和三角测量的公式。
	\item 详细讲述ORB-SLAM2方案中ORB特征点的定义与提取方法,分析其主要进程和主要功能函数;讲述CCM-SLAM方案中客户端与服务器的设计。
	\item 进行地图融合方案的验证与设计,分为理论设计和代码实现。
	\item 讲述了如何在gazebo平台通过launch文件配置仿真环境,并进行了单机和多机的视觉SLAM仿真。在单机SLAM仿真中,使用了新的Offboard控制程序,并且设计了从SLAM解算出的位姿到MAVROS消息类型位姿的转换方法;在多机SLAM仿真中,详细阐述了多机sdf文件生成模型的原理,提出了配置模型的方法,进行了编队处理,以及最后的仿真实验。
	\item 进行了真机实验和地图融合实验,分析评估了实验结果。简单介绍了真机的配置方法;主要介绍了用SLAM系统分析MP4文件并进行建图的方法。
\end{enumerate}

\section{对未来工作的展望}

根据本文的分析,视觉SLAM的算法上已经有很多优秀的方案,但是在多设备利用视觉SLAM的定位和建图信息完成定位、导航等其他任务上还有很多发展余地:

\begin{enumerate}
	\item 
	针对集群的SLAM方案还比较少,其中的通信机制还比较复杂;多机地图的融合方法还存在同一的路标点无法识别导致的重复建图问题。
	\item 单目ORB-SLAM2方法中,当无人机速度快时容易跟踪丢失且重定位失败的问题,这是没有IMU信息的视觉SLAM方法的问题之一;其方向是融合IMU信息,比如VINS方案和ORB-SLAM3方案,跟踪性能更好。
	\item 
	仿真环境的配置,由于局限于电脑的配置,不能使用真正意义上集群类型的仿真,未来有望在满足配置的电脑上进行更多无人机的仿真。
	\item 
	使用USB3.0外接Linux系统进行MP4转rosbag时,由于rosbag包会占据很大磁盘带宽,而USB3.0又不能满足,会导致实际转换得到的rosbag视频帧率变低、总时长变长;未来尽量使用直接固态硬盘或双系统的方式,解决带宽问题。
	\item 
	真机的飞行由于安全性和可操作度的限制,没有使用离板模式控制下的真机飞行;也没有对SLAM得到的路标点和地图作进一步处理,因此也没有无人机的避障和路径规划内容,没有一个自主的对未知环境的探索策略,这些都是未来的研究方向。
	
\end{enumerate}

