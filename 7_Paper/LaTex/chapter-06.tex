
\renewcommand{\baselinestretch}{1.5}
\fontsize{12pt}{13pt}\selectfont

\chapter{总结与展望} \label{conclusion}


\section{全文总结}

SLAM技术是当今机器人以及无人系统在进入未知环境时进行运动决策和场景感知的关键技术,而可协同的SLAM方案则为集群机器人或无人系统提供了更多可能。本文主要研究了SLAM技术的原理及地图拼合的技术,基于gazebo仿真平台实现对单机和多机的视觉SLAM软件在环仿真。本文主要做的工作如下:

\begin{enumerate}
	\item 在理论上,本文剖析了ROS整体框架、话题和服务两种通信机制和gazebo仿真的概念及配置。研究了PX4 AutoPilot软件的使用,其Failsafe机制和EKF2的更改以及飞行模式,使用MAVROS进行通信的离板程序控制。了解了SLAM系统,根据相机成像原理,对其内参外参矩阵进行推导。研究了ORB-SLAM2方案中ORB特征点的定义与提取方法,分析其主要进程和主要功能函数。分析了CCM-SLAM方案中客户端与服务器的设计。
	
	\item 在仿真上,讲述了如何在gazebo平台通过launch文件配置仿真环境。实现了航路点控制和编队控制飞行的方法,开发了键盘控制的速度飞行的方法。在单机SLAM仿真中,完成了无人机在离板模式下的起飞与降落,航路点飞行;使用了新的Offboard控制程序,并且设计了从SLAM解算出的位姿到MAVROS消息类型位姿的转换方法;最终完成了基于视觉定位和速度控制的的SLAM仿真,并且获得了回环的结果。在多机SLAM仿真中,详细阐述了多机sdf文件生成模型的原理,提出了配置模型的方法,配置了具有双目和单目相机的无人机;完成了多机编队控制和航路点飞行,并且最终在一定控制策略下完成了多机的SLAM仿真。
	
	\item 在实验上,主要从真实相机、数据集、录制视频三个方面完成了实验的验证。其中使用T265相机完成了校园教学楼中的SLAM实验,获得了良好的回环结果。使用EUROC数据集,验证了双机飞行的协同SLAM方法。使用在校园场景内录制的视频,并将其转换成rosbag格式,最终获得了较为良好的地图拼合场景。
\end{enumerate}

\section{对未来工作的展望}

根据本文的分析,视觉SLAM的算法上已经有很多优秀的方案,但是在多设备利用视觉SLAM的定位和建图信息完成定位、导航等其他任务上还有很多发展余地:

\begin{enumerate}
	\item
	在理论上,针对集群的SLAM方案还比较少,其中的通信机制还比较复杂;多机地图的融合方法还存在同一的路标点无法识别导致的重复建图问题。单目ORB-SLAM2方法中,当无人机速度快时容易跟踪丢失且重定位失败的问题,这是没有IMU信息的视觉SLAM方法的问题之一;其方向是融合IMU信息,比如VINS方案和ORB-SLAM3方案,跟踪性能更好;有望开发融合UWB信息的,更加稳定的多设备协同SLAM系统,或是使用多传感器融合的SLAM方案。
	\item 
	在仿真上,倾向于使用UI界面更好,拥有更加复杂仿真能力的仿真平台;可以在Linux端完成开发,在Windows端的UE4和Airsim上完成仿真;有望在仿真中实现UWB模块和激光雷达等模块,借此实现更为复杂的仿真。
	\item 
	在实验上,本实验并没有使用真正意义上的离板模式多机SLAM,有望在研究多机的稳定飞行控制之后,使用实机进行多机编队飞行;无人机的飞行没有自主能力,有望在未来的研究中配备导航算法,使无人机能够自主地完成陌生场景的定位与建图,最终完成导航,拥有自主执行任务的能力。
	
\end{enumerate}

