
\chapter*{毕业设计小结}
\addcontentsline{toc}{chapter}{毕业设计小结}

本次毕业设计不仅仅是对我大学四年来学习知识的一个总结,更是对我知识学习能力一次拓展与提高的过程。本次毕设主要以四旋翼无人机飞行控制为核心,学习了如何通过C++编程实现坐标系的转换,旋转矩阵的运算。实践了通过编程实现半物理仿真的过程,以及提高了如何将数学公式转换为编程语言的能力。同时,了解了FlightGear飞行模拟软件的通信过程,学习如何实现UDP的通信编程。因为FlightGear是一款功能强大的开源软件,更适合在Linux操作系统下进行使用,所以,学习了如何在Liunx操作系统下进行FlightGear软件的安装,启动及Shell文件的编写。在Linux系统下,更方便的使用FlightGear飞行模拟软件与Joystick摇杆结合,实现对四旋翼无人机的飞行控制。同时,本次毕设也对固定翼无人机的飞行控制进行了学习,通过FlightGear自带的JSBsim飞行动力学模型,结合Joystick飞行摇杆,通过TCP/IP编程实现对FlightGear的通信过程,完成对固定翼无人机的飞行控制。

本次毕业设计中印象比较深刻的有飞行控制过程中,坐标系转换的问题,必须保证坐标系选择的一致性,学习了如何通过C++编程实现对向量矩阵的运算过程。同时,通过对FlightGear通信模块的学习,了解了UDP, TCP/IP协议,通过C++编程实现Joystick飞行摇杆数据到FlightGear通信的过程。对六自由度非线性飞行动力学方程的推导过程,进行了十分深刻的了解。同时,对如何将数学公式变成编程语言的编程能力,有了一定提高。

经历了本次毕设,我对无人机飞行控制的过程有了一定了解,对Linux操作系统有了一定认识。在今后的学习过程中,会更加深入的学习飞行控制理论,提高自己的编程水平,熟练使用FlightGear飞行模拟软件,实现功能更强大的飞行仿真。


