
\chapter*{毕业设计小结}
\addcontentsline{toc}{chapter}{毕业设计小结}

本次毕业设计与飞控专业的相关度可能不高,只是使用了飞控的硬件和软件,也不要求我掌握具体的动力学模型和控制律。但本次毕设注重于崭新的机器人方向和计算机视觉技术(主要是视觉SLAM),注重于对编程能力的提升。虽然以XTDrone的仿真平台为样本,但是每一个launch文件和更改的SLAM代码都出自自己之手,在不断的尝试和分析过程中开始对ROS环境下的仿真有了更加深层的了解。

本次毕设的难点集中在仿真环境的底层机制和SLAM方案上,尤以SLAM方案为主。无论是ORB-SLAM2还是CCM-SLAM,都是十分优秀的开源方案。但面对一套完整的现代的SLAM方案时,一开始会无从下手,面对数量众多的源代码和头文件,以及源代码中众多类的众多函数,会难以分析该程序的数据流。但是经历这次毕设,让我逐渐掌握了针对这种大型方案快速学习的方法,这一点对日后遇到类似的方案并进行二次开发时会有很大帮助。

经历了这次毕设,我对SLAM技术有了更深的了解,对Linux操作系统也比较熟悉,对ROS的相关用法有了初步认识,提高了自己对PX4的debug能力。

